\documentclass{article}
\usepackage[margin=0.8in]{geometry}
\usepackage{titlesec}
\usepackage{amsmath}
\usepackage{booktabs}
\usepackage{hyperref}
\usepackage{tcolorbox}
\usepackage{xcolor}
\usepackage{times}
\usepackage{setspace}
\usepackage{url}

% Configure URL breaking
\urlstyle{same}
\renewcommand{\UrlBreaks}{\do\/\do-\do_\do\.\do\~\do\:\do\=}

% Increase font size
\usepackage[fontsize=11pt]{scrextend}

% Increase line spacing
\onehalfspacing

\title{What is the Value of Daily 1 MW Peak Demand Reduction for Tata Power?}
\author{SRI-B Energy Task Force}
\date{\today}

\begin{document}
\maketitle

% Summary boxes before executive summary
\begin{tcolorbox}[colback=blue!10!white, colframe=blue!50!black, title=Financial Value, fonttitle=\bfseries]
\large\centering\textbf{1 MW = INR 2.02--2.04 crore/year}\\
\normalsize in direct annual savings for Tata Power
\end{tcolorbox}

\begin{tcolorbox}[colback=green!10!white, colframe=green!50!black, title=Environmental Impact, fonttitle=\bfseries]
\large\centering\textbf{1 MW = 1,860 tonnes CO\textsubscript{2}/year}\\
\normalsize Carbon credit value: INR 46.5 lakh/year
\end{tcolorbox}

\begin{tcolorbox}[colback=orange!10!white, colframe=orange!50!black, title=Infrastructure Value, fonttitle=\bfseries]
\large\centering\textbf{1 MW = INR 4.5--6 crore}\\
\normalsize in deferred generation infrastructure costs
\end{tcolorbox}

\section*{Executive Summary}
For Tata Power, 1 MW of peak demand reduction delivers INR 2.02--2.04 crore in annual direct savings plus INR 4.5--6 crore in deferred infrastructure costs. This analysis combines operational data, regulatory filings, and demand response economics.

\section{Cost Avoidance in Generation Infrastructure}
\subsection{Deferred Capacity Expansion}
Peak reduction avoids INR 4.5--6 crore/MW capital expenditure for new coal plants:
\begin{equation}
\text{CapEx Savings} = 1\,\text{MW} \times (INR 4.5\,\text{--}\,6\,\text{crore/MW})
\end{equation}

\subsection{Fuel Cost Savings}
Daily savings from reduced peaker plant usage:
\begin{equation}
\text{Daily Savings} = 1\,\text{MW} \times 4\,\text{h} \times INR 10/\text{kWh} = INR 40,000
\end{equation}
Annual savings reach \textbf{INR 1.46 crore}.

\section{Transmission Optimization}
\subsection{Infrastructure Deferral}
Tata Power's Mumbai network saves:
\begin{equation}
\text{NPV} = \sum_{t=1}^{10} \frac{INR 1.5\,\text{crore}}{(1+0.07)^t} = INR 9.8\,\text{crore}
\end{equation}

\subsection{Loss Reduction}
Technical loss savings calculation:
\begin{equation}
\text{Annual Savings} = 94.9\,\text{MWh} \times INR 5.2/\text{kWh} = INR 4.93\,\text{lakh}
\end{equation}

\section{Demand Response Economics}
\begin{table}[ht]
\centering
\caption{Demand Response Program Economics}
\begin{tabular}{@{}lrr@{}}
\toprule
Component & Cost (INR  lakh) & Savings (INR  lakh) \\
\midrule
Customer Incentives & 12.0 & -- \\
Fuel Cost Savings & -- & 146.0 \\
T\&D Loss Reduction & -- & 4.9 \\
\bottomrule
\end{tabular}
\label{tab:dresp}
\end{table}

\section{Regulatory Benefits}
\subsection{Capacity Markets}
\begin{equation}
\text{Revenue} = 1\,\text{MW} \times INR 6\,\text{lakh/MW} = INR 6\,\text{lakh}
\end{equation}

\subsection{Renewable Credits}
\begin{equation}
\text{RPO Value} = INR 2.3\,\text{lakh/MW} \times 1\,\text{MW} = INR 2.3\,\text{lakh}
\end{equation}

\section{Environmental Impact}
Carbon credit valuation:
\begin{equation}
\text{CO}_2\,\text{Savings} = 1,860\,\text{t} \times INR 2,500/\text{t} = INR 46.5\,\text{lakh}
\end{equation}

\section*{Conclusion}
The total quantified value ranges demonstrate peak reduction's strategic importance:
\begin{equation}
\text{Total Annual Value} = INR 2.02\,\text{--}\,2.04\,\text{crore}
\end{equation}

\section*{Where did we get these numbers from?}

\textit{(Note: these are shortened-links for readability. Clicking them on the PDF copy will take us to the full link)}

\begin{enumerate}
    \item We took the data about Peak demand projections, infrastructure costs, and technical loss benchmarks from National Electricity Plan, India\\
    \textbf{Link:} \href{https://cea.nic.in/ceadocument/national-electricity-plan-vol-1-generation-notified-vide-extra-ordinary-gazette-no-1871-si-no-121-under-part-iii-section-iv-dated-28-03-2018/}{cea.nic.in/ceadocument/national-electricity-plan}
    
    \item We took the data related to Financials, renewable investments, and demand response initiatives from Tata Power Annual Report 2023-24\\
    \textbf{Link:} \href{https://tatapowertrading.com/wp-content/uploads/2023/11/ANNUAL-REPORT-2023.pdf}{tatapowertrading.com/ANNUAL-REPORT-2023.pdf}
    
    \item T\&D upgrade costs data was taken from PowerGrid of India (PGCIL)\\
    \textbf{Link:} \href{https://www.powergrid.in/sites/default/files/inline-files/Application-for-Charges.pdf}{powergrid.in/Application-for-Charges.pdf}
    
    \item Coal Plant Capital Cost data from IEA\\
    \textbf{Link:} \href{https://www.iea.org/reports/coal-2024}{iea.org/reports/coal-2024}
    
    \item AutoGrid+Tata Power Demand Response Impact report from Uplight\\
    \textbf{Link:} \href{https://uplight.com/press/tata-power-and-autogrid-expand-ai-enabled-smart-energy-management-in-mumbai-supporting-indias-net-zero-goals/}{uplight.com/press/tata-power-and-autogrid}
\end{enumerate}

\end{document}